\section{Introduction}
\label{sec:intro}

Eventual consistency facilitates high availability, but can lead to
surprising anomalies that have been
well-documented~\cite{Burckhardt2014, pldi15, Session, Dynamo,
  RedBlue}. While applications can often tolerate many of these
anomalies, there are some that adversely effect the user experience,
and hence need to be avoided. For instance, a social network
application can tolerate out-of-order delivery of unrelated posts, but
causally related posts need to be delivered in causal order;
\emph{e.g.,}, a comment cannot be delivered before the post
itself. The view count of a video on Youtube need not necessarily
reflect the precise count of the number of views, but it should not
appear to be decreasing. A bank account application may not always
show the accurate balance in an account, but neither should it let the
balance go below zero, nor should it admit operations that would
lead it to display a negative balance.

Bare eventual consistency is often too weak to ensure such high-level
application invariants; stronger consistency guarantees are needed. To
help applications enforce such high-level invariants, off-the-shelf
replicated data stores, such as Cassandra and Riak, offer tunable
consistency levels on a per-operation basis: applications can specify
the consistency level for every read and write operation they perform
on the data store. However, consistency levels offered by these
off-the-shelf stores are often defined at a very low-level. For
example, consistency levels in Cassandra and Riak assume the values of
\C{ONE}, \C{TWO}, \C{QUORUM}, \C{ALL} etc., desribing how many
physically distributed comprising the store must respond before a read
or write operation is declared successful.  It is not immediately
apparent what permutation of these low-level consistency guarantees
would let the application enforce its high-level level invariants. For
instance, what should be the consistency level of reads and writes to
the \C{posts} table\footnote{We use the word \emph{table} as an
  all-encompassing term for various key-value abstractions provided by
  data stores.} to guarantee causal order delivery of posts in the
aforementioned social network application?

Furthermore, the semantics of low-level consistency guarantees are not
uniform across different store implementations. For instance, while
\C{QUORUM} means \emph{strict quorum} (i.e., Lamport's
quorum~\cite{LamportQuorum}) in the case of Cassandra, it means a
\emph{sloppy quorum}~\cite{Dynamo} in Riak. Complicating matters yet
further, consistency semantics is often imprecisely, or even
inaccurately, defined in the informal vendor-hosted documentations.
For instance, Datastax's Cassandra documentation~\cite{dxlwt} claims
that one can achieve ``strong consistency'' with ``quorum reads and
writes'' in Cassandra.  While this claim appears reasonable
superficially (because a pair of quorum operations are serialized at
one node, at least), it is incomplete, at best, and inaccurate at
worst.\footnote{The devil is in the details of the timestamp-based
  last-writer-wins conflict resolution strategy in Cassandra, which
  need not necessarily pick the last writer due to inevitable clock
  drift across nodes.~\cite{TyconCassandra} and~\cite{JepsenCassandra}
  present counterexamples and a more accurate claim.}  Another example
of a low-level consistency enforcement construct with vaguely defined
semantics is Cassandra's Compare-and-Set ({\sc cas}) operation, which
is advertised as a ``lightweight transaction'' and exposed as a
conditional write query (eg., \C{INSERT INTO users VALUES … IF NOT
  EXISTS}).  The addition of {\sc cas} to Cassandra was coupled with
the introduction of a new consistency level named \C{SERIAL}.
Surprisingly, \C{SERIAL} is not a valid query-level consistency
parameter for a write (conditional or not), while the others
(\emph{e.g.}, \C{ONE}) are valid.\footnote{Given the advertised use
  cases for lightweight transactions (such as maintaining uniqueness
  of usernames), one might expect a {\sc cas} to be \C{SERIAL} by
  default. It is therefore unintuitive that {\sc cas} accepts a
  consistency parameter, at least to the developers of cassandra-cql,
  a popular Haskell library for programming with Cassandra, whose API
  for {\sc cas} operation incorrectly hardcodes the parameter to
  \C{SERIAL}. This bug has been reported and fixed.}  Furthermore,
Cassandra accepts a new \emph{protocol-level} consistency parameter
for a {\sc CAS} operation that can be set to \C{SERIAL}, but its
informal description doesn't explain how this parameter interacts with
the query-level consistency parameter.  The only way to unravel this
complexity is to understand low-level details of the operator's
underlying Paxos-based implementation.  Mired in this quagmire of
low-level implementation details, it is easy to lose track of our
original goal - ensuring the high-level semantics guarantees required
by the application are met as efficiently as possible by the
implementation.

In this paper, we describe \name, a declarative programming framework
for eventually consistent data stores that was built to address the
issues discussed above. \name can be realized as a thin layer on top
of any off-the-shelf eventually consistent key-value store, and as
such, provides a uniform implementation-independent interface to the
data store. \name programmers reason in terms of an abstract system
model of an eventually consistent data store (ECDS), and any
functionality offered by the store in addition to bare eventual
consistency, including stronger consistency guarantees, transactions
with tunable isolation levels etc., is required to have a well-defined
semantics in the context of this abstract model. We show that various
high-level consistency guarantees (eg., causal consistency) and
various well-known isolation levels for transactions (eg., read
committed) indeed enjoy such properties.  \name is additionally
equipped with an expressive specification language that lets data
store developers succintly describe the semantics of the functionality
they offer. A similar specification language is exposed to application
programmers, who can declare the consistency requirements of their
application as axiomatic specifications. Specifications are
constructed using primitive consistency relations such as
\emph{visibility} and \emph{session order} along with standard logical
and relational operators. A novel aspect of \name is that it can
directly compare the specifications written by application programmers
and data store developers since both are written within the same
specification language, and can thus automatically map application
requirements to the appropriate store-level guarantees. Consequently,
\name programmers can write portable code that automatically adapts to
any data store that can express its functionality in terms of \name's
abstract system model.

Another key advantage of \name is that it allows the addition of new
replicated data types (RDTs) to the store, which obviates the need to
support data types with application-specific semantics at the store
level. In addition, \name treats the convergence semantics (i.e.,
\emph{how} conflicting updates are resolved) of the data type
separately from its consistency properties (i.e., \emph{when} updates
become visible).  This separation of concerns permits
\emph{operational} reasoning for conflict resolution, and
\emph{declarative} reasoning for consistency.  The combination of
these techniques enhances overall programmability and simplifies
reasoning about application correctness.

% Rest of this paper is organized as following. The next section
% describes the abstract system model. The following section describes a
% sample application to demonstrate the style of programming promoted by
% \name. We also cover the anomalies exhibited by the application under
% eventual consistency, which highlights the need for stronger
% consistency guarantees with well-defined high-level semantics.
% ~\S~\ref{sec:language} introduces \name's specification language, and
% presents the specifications of various high-level consistency
% guarantees. 
