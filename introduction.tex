\section{Introduction}
\label{sec:intro}

Eventual consistency facilitates high availability, but eventually
consistent replication leads to anomalies that have been
well-documented~\cite{Burckhardt2014, pldi15, Session, Dynamo,
RedBlue}. While applications are willing to tolerate most of these
anomalies, there are some that adversely effect the user experience,
hence need to be avoided. For instance, a social network application
can tolerate out-of-order delivery of unrelated posts, but causally
related posts need to be delivered in the causal order; a comment
cannot be delivered before the post itself. The view count of a video
on Youtube need not necessarily reflect the actual count, but it
shouldn't appear to be decreasing. A bank account application may not
always show the accurate balance in an account, but neither should it
let the balance go below zero, nor should it display a negative
balance. Bare eventual consistency is often too weak to ensure such
high-level application invariants; stronger consistency guarantees are
needed. To help applications enforce such high-level invariants,
off-the-shelf replicated data stores, such as Cassandra and Riak,
offer tunable consistency levels on per-operation basis: applications
can specify the consistency level for every read and write operation
they perform on the data store. However, consistency levels offered by
the off-the-shelf stores are often defined at a very low level. For
example, consistency levels in Cassandra and Riak assume the values of
\C{ONE}, \C{TWO}, \C{QUORUM}, \C{ALL} etc., desribing how many nodes
in the distributed system of the store must respond before a read or
write operation is declared success. It is not immediately apparent
what permutation of these low-level consistency guarantees would let
the application enforce its high-level level invariants. For instance,
what should be the consistency level of reads and writes to the
\C{posts} table\footnote{We use the word \emph{table} as an
all-encompassing term for various key-value abstractions provided by
data stores.} so as to guarantee the causal order delivery of posts in
the aforementioned social network application? 

Furthermore, the semantics of low-level consistency guarantees are not
uniform across stores. For instance, while \C{QUORUM} means
\emph{strict quorum} (i.e., Lamport's quorum~\cite{LamportQuorum}) in
case of Cassandra, it means a \emph{sloppy quorum}~\cite{Dynamo} in
Riak. Complicating the matters further, the semantics are often
imprecisely, or even inaccurately, defined in the informal
vendor-hosted documentations of data stores. For instance, Datastax's
Cassandra documentation~\cite{dxlwt} claims that one can achieve
``strong consistency'' with ``quorum reads and writes'' in Cassandra.
While this claim appears reasonable superficially (because a pair of
quorum operations are serialized at one node, at least), it is
incomplete, at best, and inaccurate otherwise\footnote{The devil is in
the details of the timestamp-based last-writer-wins conflict
resolution strategy in Cassandra, which need not necessarily pick the
last writer due to the inevitable clock drift across
nodes.~\cite{TyconCassandra} and~\cite{JepsenCassandra} present
counterexamples and a more accurate claim.}. Another example of a
low-level consistency enforcement construct with vaguely defined
semantics is Cassandra's Compare-and-Set (CAS) operation, which is
advertised as a ``lightweight transaction'' and exposed as a
conditional write query (eg., \C{INSERT INTO users VALUES … IF NOT
EXISTS}).  Addition of CAS to Cassandra was coupled with the
introduction of a new consistency level named \C{SERIAL}.  Strangely,
\C{SERIAL} is not a valid query-level consistency parameter for a
write (conditional or not), while the rest (eg., \C{ONE}) are
valid\footnote{Given the advertised use cases for lightweight
transactions (such as maintaining uniqueness of usernames), one might
expect a CAS to be SERIAL by default. It is therefore unintuitive that
CAS accepts a consistency parameter, at least to the developers of
cassandra-cql, a popular Haskell library for programming with
Cassandra, whose API for CAS operation incorrectly hardcodes the
parameter to SERIAL. This bug has been reported and fixed.}.
Furthermore, Cassandra accepts a new \emph{protocol-level} consistency
parameter for a CAS operation (can be set to SERIAL), and its informal
description doesn't explain how this parameter interacts with the
query-level consistency parameter.  The only way to unravel this
complexity is to understand the nitty-gritty of the Paxos-based
implementation of CAS in Cassandra. In this quagmire of low-level
implementation details, it is easy to lose track of the original
intent: to obtain the high-level guarantees required by the
application.  

In this paper, we describe \name, a declarative programming framework
for eventually consistent data stores that was build to address the
issues discussed above. \name can be realized as a thin layer on top
of any off-the-shelf eventually consistent key-value store, and as
such provides a uniform implementation-independent interface to the
store. \name programmers reason in terms of an abstract model of the
ECDS, and any functionality offered by the store in addition to bare
eventual consistency, including stronger consistency guarantees,
transactions with tunable isolation levels etc., is required to have
well-defined semantics in the abstract model. We show that various
high-level consistency guarantees (eg., causal consistency) and
various well-known isolation levels for transactions (eg., read
committed) indeed have well-defined semantics in the abstract model.
\name is equipped with an expressive specification language that lets
data store developers succintly describe the semantics of the
functionality they offer. A similar specification language is exposed
to the application programmers, who declare the consistency
requirements of their application as specifications. Specifications
are constructed using primitive consistency relations such as
\emph{visibility} and \emph{session order} along with standard logical
and relational operators. A novel aspect of \name is that it can
compare the specifications written by application programmers and data
store developers, and automatically map application's requirements to
the appropriate store-level guarantees. Consequently, \name
programmers can write portable code that automatically adapts to any
data store that exposes its functionality via \name. 

Another key advantage of \name is that it allows the addition of new
replicated data types (RDTs) to the store, which obviates the need to
support data types with application-specific semantics at the store
level. In addition, \name treats the convergence semantics (i.e.,
\emph{how} conflicting updates are resolved) of the data type
separately from its consistency properties (i.e., \emph{when} updates
become visible).  This separation of concerns permits
\emph{operational} reasoning for conflict resolution, and
\emph{declarative} reasoning for consistency.  The combination of
these techniques enhances the programmability of the store.

% Rest of this paper is organized as following. The next section
% describes the abstract system model. The following section describes a
% sample application to demonstrate the style of programming promoted by
% \name. We also cover the anomalies exhibited by the application under
% eventual consistency, which highlights the need for stronger
% consistency guarantees with well-defined high-level semantics.
% ~\S~\ref{sec:language} introduces \name's specification language, and
% presents the specifications of various high-level consistency
% guarantees. 
