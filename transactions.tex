\section{Transactions}
\label{sec:transactions}

Real-world applications often need to perform atomic operations that
span multiple objects. An example is a \C{transfer} operation on bank
accounts, which needs to perform a \C{withdraw} operation on one bank
account, and a \C{deposit} operation on another. Transactions are
usually the preferred means to compose sets of operations into a
single atomic operation. However, a classical ACID transaction model
requires inter-replica coordination, leading to the loss of
availability. To address this problem, several recent
systems~\cite{Walter,Burckhardt2012,BailisHAT} have proposed
\emph{coordination-free transactions} that offer atomicity, remain
available under network partitions, but only provide weaker isolation
guarantees. Several variants of coordination-free transactions have
subtly different isolation semantics and widely varying runtime
overheads.  Fortunately, the semantics of a large subset of such
transactions can be captured elegantly in the abstract model of \name.
Towards this end, we extend the contract language with a new term
under quantifier-free propositions - ${\small \txnZ}~S_1~S_2$, where
$S_1$ and $S_2$ are sets of effects, and which introduces a new primitive
equivalence relation $\small \sametxnZ$ that holds for effects from
the same transaction. $\small \txn{a,b}{c,d}$ is just syntactic sugar
for $\small \sametxn{a}{b} ~\wedge~ \sametxn{c}{d} ~\wedge~
\neg\sametxn{a}{c}$, where $a$ and $b$ considered to belong to the
\emph{current} transaction. Since atomicity is a defining
characteristic of a transaction, we extend our formal model with the
following axiom, which guarantees that a transaction is visible in its
entirety, or it is not visible at all:

\begin{cmathpar}
\forall a,b,c.~\txn{a,b}{c} ~\wedge~ \sameobj{a}{b} ~\wedge~ \vis{a}{c}
\Rightarrow \vis{b}{c}
\end{cmathpar}

\subsection{Isolation Requirements}

\S~\ref{sec:anomalies} demonstrates how the consistency requirements of
a counter RDT's \C{read} operation can be expressed in \name. In a
similar manner, \name's specification language allows applications to
declare isolation requirements for their transactions.

Consider an implementation of a bank account RDT in \name with three
operations -- \rcf{withdraw}, \rcf{deposit}, and \rcf{getBalance},
each with straightforward semantics. Additionally, we define two
transactions -- \cf{save(amt)}, which transfers \cf{amt} from current
($c$) to savings ($s$), and \cf{totalBalance}, which returns the sum
of the balances of individual accounts. Our goal is to ensure that
\cf{totalBalance} returns the result obtained from a consistent
snapshot of the object states. The \name code for these transactions
is given below:

\noindent \begin{minipage}[t]{0.53\columnwidth}
\begin{codehaskell}
save amt = atomically do
    b <- withdraw c amt
    -- b is true iff withdraw succeeds.
    when b dollar deposit s amt
\end{codehaskell}
\end{minipage}
\begin{minipage}[t]{0.47\columnwidth}
\begin{codehaskell}
totalBalance = atomically do
    b1 <- getBalance c
    b2 <- getBalance s
    return b1 + b2
\end{codehaskell}
\end{minipage}

The \cf{atomically} construct ensures that the effects of the
operations are made visible atomically. However, atomicity itself is
insufficient in this case, as it does not guarantee that both
\cf{getBalance} operations (in \cf{totalBalance}) witness the effects
of \cf{save}.  Consequently, \cf{getBalance} on \cf{s} may not
witness the \cf{deposit} on \cf{s} from \cf{save}, although
\cf{getBalance} on \cf{c} witnesses the \cf{withdraw} on \cf{c},
resulting in \cf{totalBalance} reporting an inconsistent balance.

Observe that the aforementioned anomaly can be averted by requiring
that both \cf{getBalance} operations in a \cf{totalBalance} transaction
witness the same set of \cf{save} actions. This requirement can be
captured succinctly in our specification language:
\begin{cmathpar}
\begin{array}{l}
\forall (a, b:\rcf{getBalance}),(c:\rcf{withdraw} \vee \rcf{deposit}),
(d:\rcf{withdraw} \vee \rcf{deposit}).\\
\hspace*{2in}\txn{a,b}{c,d} ~\wedge~ \vis{c}{a} ~\wedge~ \sameobj{d}{b} \Rightarrow \vis{d}{b}
\end{array}
\end{cmathpar}
\noindent The key idea in the above definition is to use the $\small
\txnZ$ primitive to relate operations on different objects performed
in a single transaction.

\subsection{Isolation Guarantees}
\label{sec:isolation-guarantees}

The isolation semantics of a transaction determines how the transaction
witnesses the effects of previously committed transactions\footnote{It
is informative to compare isolation with atomicity. While the former
constrains how the current transaction witnesses the effects of other
transactions, the later determines how other transactions witness the
effects of the current transaction.}. As mentioned previously, the
isolation semantics of many variants of coordination-free transactions
can be expressed in \name's specification language. For demonstration,
we pick three well-understood coordination-free transactions -- Read
Committed (RC)~\cite{Berenson95}, Monotonic Atomic View
(MAV)~\cite{BailisHAT} and Repeatable Read (RR)~\cite{Berenson95}, and
express their isolation semantics in \name.

ANSI RC isolation level guarantees that a transaction only witnesses
the effects of committed transaction:
\begin{smathpar}
\begin{array}{l}
\forall a,b,c.~\txn{a}{b,c} \wedge \sameobj{b}{c} \wedge~ \vis{b}{a}
\Rightarrow \vis{c}{a}
\end{array}
\end{smathpar}
Note that RC is the dual of atomicity; it can be guaranteed with no
additional effort if all transactions in the system are atomic.

MAV semantics ensures that if some operation in a transaction $T_1$
witnesses the effects of another transaction $T_2$, then subsequent
operations in $T_1$ will also witness the effects of $T_2$:
\begin{smathpar}
\begin{array}{l}
\forall a,b,c,d.~\txn{a,b}{c,d} ~\wedge~ \so{a}{b} ~\wedge~ \vis{c}{a}
~\wedge~ \sameobj{d}{b} \Rightarrow \vis{d}{b}
\end{array}
\end{smathpar}
MAV semantics is useful for maintaining the integrity of foreign key
constraints, materialized views and secondary
updates~\cite{BailisHAT}.

ANSI RR semantics requires that the transaction witness a snapshot of
the data store state. More concretely, RR requires that if an
operation in transaction $T_1$ witnesses the effects of transaction
$T_2$, then all the remaining operations should also witness the
effects of $T_2$:
\begin{smathpar}
\begin{array}{l}
\forall a,b,c,d.~\txn{a,b}{c,d} ~\wedge~ \vis{c}{a} ~\wedge~
\sameobj{d}{b} \Rightarrow \vis{d}{b}
\end{array}
\end{smathpar}
Note that this is precisely the semantics required by the
\cf{totalBalance} transaction to ensure that it returns a balance that
reflects a consistent snapshot of the data store. Hence, it is
sufficient to execute \cf{totalBalance} at RR isolation level.

The ease of reasoning with precisely stated high-level guarantees thus
extends to transactions as well. Furthermore, the ability to express
the isolation requirements of an application, along with the semantics
of various isolation guarantees provided by the store in the \name's
specification language allows us to define a classification
scheme~\cite{pldi15}, similar to the one in
\S~\ref{sec:classification-scheme}, to automatically map requirements
to guarantees.
